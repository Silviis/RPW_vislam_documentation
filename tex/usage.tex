\chapter{Usage}\label{ch:usage}
This chapter walks through the process of setting up and using the repository and the containerization environment to launch the SLAM system.

\section{Cloning the repository and setting up the environment}
To clone the repository and its dependencies run the following commands:
\begin{minted}{bash}
$ cd ~/
$ git clone --recurse-submodules git@github.com:Silviis/RPW_ros2_ws.git
\end{minted}

The repository may be used within native ROS2 Humble environment without Docker containers. If that is the scenario you can skip the containerization parts of this document.

\section{Usage of the repository \& Docker container environment }
This section covers the basic usage of the repository and container environment. All of the commands in this section assume that the current working directory is inside the repository:
\begin{minted}{bash}
$ cd ~/RPW_ros2_ws
\end{minted}


To launch the containers:
\begin{minted}{bash}
$ docker compose up -d
\end{minted}

To shutdown the containers:
\begin{minted}{bash}
$ docker compose down
\end{minted}

To get a bash shell running inside the container:
\begin{minted}{bash}
$ docker compose exec ros2_cam bash
\end{minted}

To rebuild the containers:
\begin{minted}{bash}
$ docker compose build
\end{minted}

You can omit the \textbf{-d} flag if you want to attach the containers stdout to the current terminal.

