\chapter{Introduction}\label{ch:introduction}
VI-SLAM (Visual Inertial Simultaneous Localization and Mapping) is a method for real time state estimation and global mapping that utilizes both inertial and visual measurements. Fusing stereo or monocular image frames with high-rate inertial data allows for utilization of the best parts of both: rich geometric information from the camera(s) and short-term motion dynamics and scale ambiguity resolving from the IMU (Inertial Measurement Unit). 

Goal of the project was to run a full, functional real-time SLAM on Jetson Xavier, using a custom-built sensor box with an IMX219-83 stereo camera and an IMU. Second aim of the project was to create a reproductible, containerized software pipeline which enables reliable deployment, calibration and visualization of the SLAM results on embedded hardware. 

 