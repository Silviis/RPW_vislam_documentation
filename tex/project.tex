\chapter{Project Organization and Management}

\section{Team Contributions}
% who did what
\textbf{Juuso} was primarily responsible for camera–IMU intrinsic and extrinsic calibration, the
development of the stereo camera pipeline, image acquisition and rectification, IMU data
acquisition, and configuration of the visual-inertial SLAM system. Juuso also contributed
to technical documentation related to sensor setup and calibration.

\textbf{Joona} was responsible for technical documentation and project reporting.

\textbf{Eetu} handled software integration, Jetson Xavier NX environment setup, development of the
containerized CI/CD pipeline, and overall system architecture design. Eetu also contributed
to technical documentation and collaborated closely with Juuso on the stereo camera pipeline
and visual-inertial SLAM configuration to ensure consistent integration across system
components.
 
\section{Project Timeline and Management}
% how the project was planned and executed
At the beginning of the project, a Gantt chart was created to support scheduling and overall
project management. The project plan was divided into two main phases: implementation of the
visual-inertial SLAM system and experimental evaluation. Each phase was further subdivided
into smaller tasks to clarify dependencies and estimate the required workload. The original
Gantt chart is shown in Figure~\ref{fig:RPW_VSLAM_project.png}.

\addpicture{RPW_VSLAM_project.png}{Original project plan illustrated using a Gantt chart.}

Despite careful planning, the project did not fully follow the initial schedule. The main
reasons for this were unanticipated technical challenges encountered during system
integration. In particular, significant effort was required to establish a reliable stereo
camera pipeline and to perform accurate sensor calibration. These tasks proved more complex
and time-consuming than originally estimated.

Since successful sensor data acquisition was a prerequisite for most downstream components,
many tasks could not be parallelized as planned. Delays in the camera pipeline therefore
propagated to later stages of the project, including SLAM configuration and experimental
evaluation.

In addition, work on the project tended to occur in larger focused sessions rather than as
small, continuous increments. While this approach allowed rapid progress once major issues
were resolved, it also reduced the flexibility to adjust to unexpected delays. These
experiences highlight the importance of allocating sufficient time for debugging and
integration when working with complex hardware–software systems, particularly on embedded
platforms.