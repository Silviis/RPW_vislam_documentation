\chapter{Project Organization and Management}

\section{Team Contributions}
% who did what
Juuso did the camera-IMU intrinsic and extrinsic calibration, the stereo camera pipeline, image acquisition and rectification, Imu data acquisition and finally VI-Slam configuration. 

Joona was responsible for technical documentation and project reporting. 

Eetu took care of the software integration, Jetson Xavier NX environment setup, containerized CI/CD pipeline, system architecture design and also worked with Juuso on the stereo camera pipeline and VI-SLAM configuration. 

 
\section{Project Timeline and Management}
% how the project was planned and executed
At the start of the project, a Gantt chart was created to help with scheduling and project management. The Gantt graph has two main sections, setting up the VI-SLAM and then testing it. Both sections were furthermore split into sub-sections to more easily identify and quantify the needed tasks. 

\begin{figure}[htp]
    \centering
    \includegraphics[width=14cm]{figures/RPW_VSLAM_project.png}
    \caption{Original plans}
    \label{fig:gantt}
\end{figure}

 Unfortunately, we could not quite keep up with the planned schedule. The main reason for this was different technical difficulties. A lot more time had to be spent debugging problems than was originally anticipated, mostly regarding getting the visual feed out from the camera and calibrating it. Since getting the camera data was necessary for the other steps, we could not really parallelize tasks either. Also, rather than working on the project consistently little by little, most of the work was done in larger batches, usually right before the deadlines. 