\chapter{Conclusion and Future Work}\label{sec:conclusion}

This project demonstrated the successful implementation of a real-time visual-inertial
SLAM system on the Jetson Xavier NX using a custom-built sensor box consisting of a stereo
camera and an IMU. The complete pipeline, from sensor data acquisition and calibration to
state estimation, mapping, and visualization, was implemented using ROS~2 and deployed
through a containerized software architecture. By leveraging automated CI/CD workflows, the
system achieved reproducible and efficient deployment despite the constraints of embedded
hardware and platform-specific compatibility limitations.

The results show that reliable visual-inertial SLAM can be achieved on resource-constrained
embedded platforms when careful attention is paid to system architecture, sensor setup, and
deployment strategy. The separation of sensor processing and SLAM computation into distinct
containers proved effective in addressing Jetson-specific software constraints while
maintaining a modular and extensible system design.

\section{Future Work}

While the system performed well in controlled indoor environments, several improvements can
be identified for future development. One significant enhancement would be the use of
hardware-synchronized stereo cameras. In the current setup, the left and right camera images
are captured sequentially in software and timestamped individually, resulting in a small
temporal offset between stereo image pairs. Although acceptable for the evaluated scenarios,
this offset can introduce inconsistencies during rapid motion or in highly dynamic scenes.

Employing cameras with hardware-level synchronization or global shutter support would ensure
more accurate temporal alignment between stereo frames. This would improve feature
correspondence, depth estimation accuracy, and overall SLAM robustness, particularly in
challenging motion conditions.

Another important direction for future work is improving the mechanical design of the
sensor enclosure for outdoor operation. The current enclosure is suitable for indoor use
but offers limited protection against environmental factors such as dust, moisture, and
direct sunlight. A more robust enclosure with improved weather sealing, vibration damping,
and thermal management would be necessary for reliable long-term outdoor deployment.

Additional future work could include extending the system to outdoor environments, improving
time synchronization between sensors, and evaluating alternative SLAM backends or sensor
configurations to further enhance performance and robustness.